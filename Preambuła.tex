%%%%%%%%%%%%%%%%%%%%%%%%%%%%%%%%%%%%%%
%% Preambuła w pliku z moją rozprawą doktorską
%%%%%%%%%%%%%%%%%%%%%%%%%%%%%%%%%%%%%%

\documentclass[oneside,12pt]{mwbk} % "mbwk" to polski odpowiednik klasy "book". Niestety, gryzie się z pakietem "bicaption".

%%%%%%%%%%%%%%%%%%%%%%%%%%%%%%%%%%%%%%
%%%% Tylko niezbędne pakiety
    \usepackage[round]{natbib}
    \usepackage{apalike}
        \let\bibhang\relax
    \usepackage{amsmath, amssymb}
    \usepackage{float}
    
    \usepackage{polski}
     \usepackage[english, polish]{babel} 
    \usepackage[T1]{fontenc}
    \usepackage[utf8]{inputenc}% nieszczęśni użytkownicy Windows powinni
                               % zmienić utf8 na cp1250
    \usepackage[unicode, naturalnames, breaklinks]{hyperref}
        \def\UrlBreaks{\do\/\do-} %% "Złamanie" zbyt długich URL-i
    
    \usepackage{longtable,booktabs}
    
    \usepackage{graphicx}
        \graphicspath{ {grafika/} {ObrazkiZRaportuZDyzagregacji/figures/} }
%%%%%%%%%%%%%%%%%%%%%%%%%%%%%%%%%%%%%%

%%%%%%%%%%%%%%%%%%%%%%%%%%%%%%%%%%%%%%
%%%% Wymagania związane z rozporządzeniem Rektora:
%%%% https://www.bip.pw.edu.pl/var/pw/storage/original/application/d68cb63fd8a9f6591f0449caab45696b.pdf

    % Paragraf 1.2 - strona tytułowa powinna być zgodna z załącznikiem nr 2:
    \usepackage{pdfpages} % do dołączenia strony tytułowej w PDF
    
    % Paragraf 1.5 - dziekan może ustalić szczególne wymagania 
    % (i ustalił - podwójne podpisy, po polsku i angielsku):
    % https://tex.stackexchange.com/questions/370134/having-better-control-on-captions-in-two-languages-under-each-image
    \usepackage[lang=english]{bicaption}
    
    % Załącznik nr 1, punkt 1 i 2 - A4. marginesy po 2,5 cm
    \usepackage[a4paper,left=2.5cm,right=2.5cm,top=2.5cm,bottom=2.5cm]{geometry}
    
    % Załącznik nr 1, punkt 3 i 4 - czcionka Times New Roman, wielkość czcionki 12 pkt.
    \usepackage{times}
    
    % Załącznik nr 1, punkt 5 - interlinia: 1,5
    %Uwaga: Jeśli chcesz używać interlinii równej 1,5 jako wartość należy wstawić "1.3". Natomiast w przypadku interlinii równej 2, wartość to "1.6" (na podstawie https://pl.wikibooks.org/wiki/LaTeX/Formatowanie)
    \linespread{1.3}
    
    % Załącznik nr 1, punkt 6 - numeracja stron - wyśrodkowana, na dole strony
    %the package fancyhdr doesn't overwrite the default styles instead it defines a new one named fancy  
    \usepackage{fancyhdr} 
    %\fancyhf{}
    %\lhead{ \chaptername \thechapter}
    \lhead{}
    \setlength{\headheight}{26.98592pt} 
    \cfoot{\thepage}

%%%%%%%%%%%%%%%%%%%%%%%%%%%%%%%%%%%%%%

%%%%%%%%%%%%%%%%%%%%%%%%%%%%%%%%%%%%%%
%%%% Zdefiniowanie środowisk do słów kluczowych i streszczenia

    \providecommand{\keywords}[1]{\textbf{\textit{Słowa kluczowe:}} #1}
    \providecommand{\keywordsANG}[1]{\textbf{\textit{Keywords:}} #1}
    
    %% TODO: Zdefiniować tylko jedno środowisko do streszczeń i słów kluczowych!!
    \newenvironment{abstract}%
        {\cleardoublepage\thispagestyle{empty}\begin{center}%
        %{\cleardoublepage\thispagestyle{empty}\null\vfill\begin{center}%
        \bfseries\abstractname\end{center}\thispagestyle{empty}}%
        {\vfill\null\thispagestyle{empty}}
        
    \newenvironment{abstract-ang}%
        {\cleardoublepage\thispagestyle{empty}\begin{center}%
        %{\cleardoublepage\thispagestyle{empty}\null\vfill\begin{center}%
        \bfseries Abstract \end{center}\thispagestyle{empty}}%
        {\vfill\null\thispagestyle{empty}}
        
%%%%%%%%%%%%%%%%%%%%%%%%%%%%%%%%%%%%%%


%%%%%%%%%%%%%%%%%%%%%%%%%%%%%%%%%%%%%%
%%%% Inne ustawienia
    \pdfstringdefDisableCommands{%
      \let\enspace\empty  % this causes the warning for \kern
      \let\noindent\empty % this causes the warning for \indent
    }
    
    % Zgodnie z: http://tex.stackexchange.com/questions/83440/inputenc-error-unicode-char-u8-not-set-up-for-use-with-latex
    % to define non-breaking space (and omit error message)
    \DeclareUnicodeCharacter{00A0}{~}
    %
    
    %cytowanie otoczone nawiasem
    \renewcommand{\cite}{\citep}
    %Zmiana nawiasu z okrągłego na kwadratowy - Selecting citation style and punctuation
    \bibpunct{[}{]}{,}{a}{}{;}
    
    %% Ten typ listy jest używany przez wyjście z Rmarkdown
    \providecommand{\tightlist}{%
      \setlength{\itemsep}{0pt}\setlength{\parskip}{0pt}}
%%%%%%%%%%%%%%%%%%%%%%%%%%%%%%%%%%%%%%  
